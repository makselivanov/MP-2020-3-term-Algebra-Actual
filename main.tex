\documentclass[10pt,a4paper,oneside]{book}
\usepackage[a4paper,includeheadfoot,top=10mm,bottom=10mm,left=10mm,right=10mm]{geometry}
\usepackage[utf8]{inputenc}
\usepackage[russian]{babel}
\usepackage{cmap} % Поддержка поиска русских слов в PDF (pdflatex)
\usepackage{comment}
%\usepackage{pdfsync} % синхронизация

\usepackage{amsmath,amsthm,amssymb,amscd,array}
\usepackage{latexsym}
\usepackage{stmaryrd} % Для знака нормальной подгруппы
\usepackage{misccorr} % российская полиграфия
\usepackage{indentfirst}% Красная строка в первом абзаце
\usepackage{ccaption} % Заголовки таблиц и рисунков
\usepackage{fancyhdr} % колонтитулы
\usepackage{hyperref} % гиперссылки

\usepackage{rotating} % Поворот текста
\usepackage{graphicx} % Вставка изображений
\usepackage{xcolor}
\usepackage{pgf}
\usepackage{tikz}
\usepackage{tikz-cd}
\usetikzlibrary{arrows,backgrounds,patterns,matrix,shapes,fit,calc,shadows,plotmarks}
\graphicspath{ {./} } % относительно main.tex

\usepackage{arydshln} % штрихованные линии в массивах
\usepackage{mathtools} % выравнивание в матрицах
\usepackage{multirow} % слияние в столбце
\usepackage{multicol} % нумерация в нескольких колонках


\newtheorem{uprz}{\color{violet!100!black} Упражнение}
\newtheorem{predl}{\color{blue!50!black} Предложение}
\newtheorem{komment}{\color{green!50!blue} Комментарий}
\newtheorem{conj}{Гипотеза}
\newtheorem{notation}{\color{yellow!30!red} Обозначение}


\theoremstyle{definition}
\newtheorem{kit}{Кит}
\newtheorem*{rem}{\color{green!50!blue}Замечание}
\newtheorem{zad}{\color{violet!100!black}Задача}
\newtheorem*{defn}{\color{yellow!30!red} Определение}
\newtheorem*{fact}{Факт}
\newtheorem{thm}{\color{red!40!black}Теорема}
\newtheorem*{thmm}{\color{red!40!black} Теорема}
\newtheorem{lem}{\color{green!50!black}Лемма}
\newtheorem{cor}{\color{green!45!black}Следствие}
\newtheorem{utvr}{\color{blue!50!black}Утверждение}


\newcommand\tikznode[3][]%
   {\tikz[remember picture,baseline=(#2.base)]
      \node[minimum size=0pt,outer sep=0pt,#1](#2){#3};%
   }
\tikzset{>=stealth}


\hypersetup{
    colorlinks,
    linkcolor={blue!50!black},
    citecolor={blue!50!black},
    urlcolor={red!80!black}
}
% цвета для ссылок


\makeatletter
\renewcommand*\env@matrix[1][*\c@MaxMatrixCols c]{%
  \hskip -\arraycolsep
  \let\@ifnextchar\new@ifnextchar
  \array{#1}}
\makeatother


\renewcommand{\leq}{\leqslant}
\renewcommand{\geq}{\geqslant}
\renewcommand{\proofname}{Доказательство}
\renewcommand{\mod}{\,\operatorname{mod}\,}
\renewcommand{\Re}{\operatorname{Re}}
\newcommand{\mf}[1]{\mathfrak{#1}}
\newcommand{\mcal}[1]{\mathcal{#1}}
\newcommand{\mb}[1]{\mathbb{#1}}
\newcommand{\mc}[1]{\mathcal{#1}}
\newcommand{\tbf}[1]{\textbf{#1}}
\newcommand{\ovl}{\overline}
\newcommand{\Spec}{\operatorname{Spec}}
\newcommand{\K}{\operatorname{K_0}}
\newcommand{\witt}{\operatorname{W}}
\newcommand{\gw}{\operatorname{GW}}
\newcommand{\coh}{\operatorname{H}}
\newcommand{\dist}{\operatorname{dist}}
\newcommand{\cl}{\operatorname{Cl}}
\newcommand{\Vol}{\operatorname{Vol}}
\newcommand\tgg{\mathop{\rm tg}\nolimits}
\newcommand\ccup{\mathop{\cup}}
\newcommand{\id}{\operatorname{id}}
\newcommand{\lcm}{\operatorname{lcm}}
\newcommand{\chr}{\operatorname{char}}
\newcommand{\rk}{\operatorname{rk}}
\DeclareMathOperator{\Coker}{Coker}
\DeclareMathOperator{\Ker}{Ker}
\newcommand{\im}{\operatorname{Im}}
\renewcommand{\Im}{\operatorname{Im}}
\newcommand{\Tr}{\operatorname{Tr}}
\newcommand{\re}{\operatorname{Re}}
\newcommand{\tr}{\operatorname{Tr}}
\newcommand{\ord}{\operatorname{ord}}
\newcommand{\Stab}{\operatorname{Stab}}
\newcommand{\orb}{\operatorname{\mathcal O}}
\newcommand{\Fix}{\operatorname{Fix}}
\newcommand{\Hom}{\operatorname{Hom}}
\newcommand{\End}{\operatorname{End}}
\newcommand{\Aut}{\operatorname{Aut}}
\newcommand{\Inn}{\operatorname{Inn}}
\newcommand{\Out}{\operatorname{Out}}
\newcommand{\GL}{\operatorname{GL}}
\newcommand{\SL}{\operatorname{SL}}
\newcommand{\SO}{\operatorname{SO}}
\renewcommand{\O}{\operatorname{O}}
\renewcommand{\U}{\operatorname{U}}
\newcommand{\Sym}{\operatorname{Sym}}
\newcommand{\Adj}{\operatorname{Adj}}
\newcommand{\Disc}{\operatorname{Disc}}
\newcommand{\cnt}{\operatorname{cont}}
\newcommand{\Frob}{\operatorname{Frob}}
\newcommand{\Iso}{\operatorname{Iso}}
\newcommand{\Isom}{\operatorname{Isom}}
\newcommand{\supp}{\operatorname{supp}}
\newcommand{\di}{\mathop{\,\scalebox{0.85}{\raisebox{-1.2pt}[0.5\height]{\vdots}}\,}}
\newcommand{\ndi}{\mathop{\not\scalebox{0.85}{\raisebox{-1.2pt}[0.5\height]{\vdots}}\,}}
\newcommand{\nequiv}{\not \equiv}
\newcommand{\Nod}{\operatorname{\text{НОД}}}
\newcommand{\Nok}{\operatorname{\text{НОК}}}
\newcommand{\sgn}{\operatorname{sgn}}
\newcommand{\codim}{\operatorname{codim}}
\newcommand{\Aff}{\operatorname{Aff}}
\newcommand{\AGL}{\operatorname{AGL}}
\newcommand{\PSL}{\operatorname{PSL}}
\newcommand{\Volume}{\operatorname{Volume}}

\def\llq{\textquotedblleft} 
\def\rrq{\textquotedblright} 
\def\exm{\noindent {\bf Примеры:}}


\def\Cb{\ovl{C}}
\def\ffi{\varphi}
\def\pa{\partial}
\def\V{\bf V}
\def\La{\Lambda}
\def\eps{\varepsilon}
\def\del{\delta}
\def\Del{\Delta}
\def\A{\EuScript{A}}
\def\lan{\left\langle }
\def\ran{\right\rangle}
\def\bar{\begin{array}}
\def\ear{\end{array}}
\def\beq{\begin{equation}}
\def\eeq{\end{equation}}
\def\thrm{\begin{thm}}
\def\ethrm{\end{thm}}
\def\dfn{\begin{defn}}
\def\edfn{\end{defn}}
\def\lm{\begin{lem}}
\def\elm{\end{lem}}
\def\zd{\begin{zad}}
\def\ezd{\end{zad}}
\def\prdl{\begin{predl}}
\def\eprdl{\end{predl}}
\def\crl{\begin{cor}}
\def\ecrl{\end{cor}}
\def\rm{\begin{rem}}
\def\erm{\end{rem}}
\def\fct{\begin{fact}}
\def\efct{\end{fact}}
\def\enm{\begin{enumerate}}
\def\eenm{\end{enumerate}}
\def\pmat{\begin{pmatrix}}
\def\epmat{\end{pmatrix}}
\def\utv{\begin{utvr}}
\def\eutv{\end{utvr}}
\def\upr{\begin{uprz}}
\def\eupr{\end{uprz}}
\def\nrml{\trianglelefteqslant}

\frenchspacing
\righthyphenmin=2
%\usepackage{floatflt}
\captiondelim{. }

\title{Современное программирование \\ 
Конспект по алгебре, 3 семестр}
\date{}


\begin{document}

\tableofcontents

\chapter{Линейная алгебра}

\section{Кватернионы}


Наша цель сейчас рассказать про геометрию трехмерного пространства используя при этом определённые алгебраические конструкции. А именно, ещё в XIX веке Уильям Роуэн Гамильтон стал искать аналогичную комплексным числам алгебраическую систему на трёхмерном пространстве.  
Однако, подходящий аналог удалось найти только в четырёхмерной ситуации.


Рассмотрим вещественное подпространство в алгебре матриц $M_2(\mb C)$ вида
$$\mb H = \left\{\pmat \alpha & \beta \\ -\ovl{\beta} & \ovl{\alpha} \epmat \right\}.$$
Базис этого пространства, как вещественного векторного пространства, состоит из матриц 
$$ 1=\pmat 1 & 0 \\ 0& 1 \epmat, i= \pmat i & 0 \\ 0& -i \epmat, j=\pmat 0& 1 \\ -1 & 0 \epmat, k=\pmat 0 & i \\ i & 0\epmat. $$ 
Покажем, что это вещественная подалгебра в $M_2(\mb C)$ и следовательно ассоциативное кольцо. 
Для этого достаточно показать, что произведение базисных снова лежит в $\mb H$. Имеем $$i^2=j^2=k^2=-1 \text{ и } ij=k=-ji,$$ откуда $$ik= iij=-j=jii=-ki \text{ и } jk=-jjk=-i=-kj.$$ Таким образом $\mb H$ образует ассоциативную алгебру размерности 4 над $\mb R$.
 
\dfn[Алгебра кватернионов] $\mb H$ называется алгеброй кватернионов. 
\edfn
Мы больше не будем думать (кроме одного нюанса) про кватернионы как про матрицы, а будем записывать их через $i,j,k$. Именно так обычно кватернионы и вводят -- как формальный суммы $a+bi+cj+dk$, для произведения которых выполнены тождества $i^2=j^2=k^2=-1$ и $ij=-ji=k$. Посмотрев на кватернионы как на матрицы мы сэкономили на доказательстве ассоциативности умножения.

\zd Алгебра кватернионов не снабжается структурой $\mb C$-алгебры.
\ezd






\dfn[Векторная и скалярная часть, сопряжённый кватернион] Пусть $x= a+bi+cj+dk$ кватернион. Определим вещественную или скалярную часть $\Re x=a$ и векторную часть $v= bi+cj+dk$ кватерниона. Сопряжённым кватернионом называется $\ovl{x}= a-bi-cj-dk= \Re x - \Im x =a-v$. 
\edfn



Посмотрим, как перемножаются кватернионы. Если оба кватерниона  $x$, $y$ разделить на скалярную и векторную части $x=a+v$, $y=b+u$ то $xy=ab+au+bv+ vu$. Нам осталось разобраться с умножением векторных частей. 

Рассмотрим произведение двух чисто мнимых кватернионов $uv=-\lan u,v\ran+[u,v]$. Его вещественная часть совпадает с минус скалярным произведением векторов. Про мнимую часть мы поговорим отдельно.

\dfn[Векторное произведение] Пусть $u,v \in \mb R^3$ два вектора. Тогда их векторным произведением называется вектор $[u,v]$.
\edfn

Если расписать в координатах $u=x_1i+x_2j+x_3k$ и  $v=y_1i+y_2j+y_3k$, то векторное произведение задаётся формулой

$$[u,v]= (x_2y_3-x_3y_2)i + (x_3y_1-x_1y_3)j + (x_1y_2- x_2y_1)k= \begin{vmatrix} i& j&k \\ x_1 & x_2 & x_3 \\ y_1 & y_2 & y_3 \end{vmatrix} $$

\rm Операция $(u,v) \to [u,v]$ является билинейной и антисимметричной, то есть $[u,u]=0$ и, следовательно, $[u,v]=-[v,u]$.
\erm

Последнее замечание позволяет нам легко вычислить $x \ovl{x}= a^2+ \lan v,v \ran$. Это приводит нас к определению:

\dfn[Норма кватерниона] Определим норму кватерниона как $$\|x\|=\sqrt{x\ovl{x}}=\sqrt{ a^2+b^2+c^2+d^2}=\sqrt{\ovl{x}x}.$$
\edfn 


Норма кватерниона, как и модуль комплексного числа всегда положительны для ненулевых элементов. Это позволяет заметить, что

\dfn[Обратный кватернион] Если $0\neq x \in \mb H$, то $x^{-1}=\frac{\ovl{x}}{\|x\|^2}$. 
\edfn

Таким образом мы получили первый (и для нас единственный) пример некоммутативного кольца с делением. Такие кольца называются телами. Напоминаю, что алгебра для нас ассоциативна и с единицей. Неассоциативные алгебры представляют интерес. Например, можно взять $\mb R^3$, где в качестве умножения взято векторное произведение. Это пример неассоциативной алгебры или, точнее, алгебры Ли. В этом курсе мы не  обсуждаем неассоциативные алгебры в связи с тем, что им обычно находится применение либо внутри физических дисциплин, либо внутри самой математики и редко где ещё. 

Какие ещё свойства есть у отображения нормы? Если следовать параллели с комплексными числами, то стоит посмотреть, что происходит с нормой произведения. Для того, чтобы не обременяться вычислениями сделаем небольшой трюк и на секунду вспомним матричное представление кватернионов. Заметим, что на матричном языке, норма -- это $\|x\|=\sqrt{\det x}$, откуда получаем

\lm[Норма мультипликативна] $\|xy\|=\|x\|\|y\|$. В частности, $\|x^{-1}\|=\|x\|^{-1}$.
\proof Вспомним в последний раз, что кватернионы задаются матрицами из $M_2(\mb C)$. Пусть 
$$x=\pmat \alpha & \beta \\ -\ovl{\beta} & \ovl{\alpha}\epmat.$$
Тогда $$\|x\|^2=|\alpha|^2+|\beta|^2=\det \pmat \alpha & \beta \\ -\ovl{\beta} & \ovl{\alpha}\epmat,$$
а определитель мультипликативен.
\endproof
\elm



Продолжим. Используя мультипликативность нормы легко доказать 
\lm Отображение $x \to \ovl{x}$ является антиизоморфизмом алгебр, то есть $\ovl{ab}=\ovl{b}\ovl{a}$.
\proof Линейной ясна. Пусть $x,y \neq 0$. Тогда $$\frac{\ovl{y}\,\ovl{x}}{\|y\|^2\|x\|^2}=y^{-1}x^{-1}=(xy)^{-1}=\frac{\ovl{xy}}{\|xy\|^2}.$$
\elm

На самом деле и здесь можно было воспользоваться матричным представлением. А именно, можно заметить, что операция сопряжения кватернионов совпадает на этом языке с транспонированием и сопряжением соответствующей комплексной матрицы. Вернёмся теперь к векторному произведению.





\lm[Свойства векторного произведения] Верны следующие свойства\\
1) Для любых $u,v \in \mb R^3$ верно $u\bot [u,v]$. Точнее $$u[u,v]= -\|u\|^2v+ \lan u,v\ran u$$
2) $\| [u,v]\|= \|u\|\|v\| \cdot |\sin \ffi |$, где $\ffi$ --  это угол между $u$ и $v$.
\elm
\proof Для того, чтобы посчитать скалярное произведение $\lan u, [u,v]\ran$ необходимо посчитать скалярную часть $u[u,v]$. 
$$u[u,v]= u (uv+ \lan u,v \ran)= u^2 v+ \lan u,v \ran u= -\|u\|^2 v+ \lan u,v \ran u$$
Последнее выражение, очевидно, чисто векторное.
Теперь
\begin{align*}
&\|[u,v]\|^2= -[u,v][u,v]= (uv + \lan u,v\ran)(vu + \lan u,v\ran)=\\
&=\|u\|^2\|v\|^2+ \lan u,v\ran^2 + \lan u,v\ran (uv+vu)=\|u\|^2\|v\|^2 - \lan u,v\ran^2= \|u\|^2\|v\|^2(1-\cos^2 \ffi)
\end{align*}
\endproof

\dfn Обозначим за $\mb H_{1}$ группу кватернионов, по норме равных единице.
\edfn

\thrm Отображение $\mb H_{1}\to \GL_3(\mb R)$ заданное по правилу $x\to (y \to xyx^{-1})$ корректно определено и даёт сюръективный  гомоморфизм из группы кватернионов единичной нормы в $\SO_3(\mb R)$. Ядро этого гомоморфизма состоит из $\{\pm 1\}$. Точнее, если единичный кватернион $x$  представим в виде $x=a+bv$, то соответствующее ему вращение есть вращение относительно  оси $\lan v \ran$ на угол $2\ffi$, где $\cos \ffi= a$, $\sin \ffi= b$ или тождественное преобразование в случае $v=\pm 1$.
\ethrm
\proof Рассмотрим преобразование $L_x \colon \mb H \to \mb H$ вида $y \to xyx^{-1}$. Прежде всего покажем, что мы получили ортогональное преобразование $\mb R^4$. Имеем
 $$\|xvx^{-1}\|=\|x\| \|v\| \|x^{-1}\| = \|v\|.$$
Теперь заметим, что преобразование $L_x$ сохраняет на месте вектор 1 и, следовательно, его ортогональное дополнение, то есть $\mb R^3$. Таким образом $L_x$ ограничивается на $\mb R^3$. Далее, очевидно, $L_xL_y= L_{xy}$. Осталось посчитать ядро гомоморфизма и явный вид отображения $L_x$. Заметим, что если $x=a+bv$, то $L_x$ оставляет $v$ на месте. Действительно, при $b\neq 0$ 
$$xbvx^{-1}=x(x-a)x^{-1}= x-a=bv.$$
Вычислим угол поворота. Для этого рассмотрим нормированный вектор  $u\bot v$ и $[u,v]$, которые образуют ортонормированный базис дополнения и посчитаем $xux^{-1}$ и $x[u,v]x^{-1}$. Из условия ортогональности следует, что $uv=[u,v]=-[v,u]=-vu$ и аналогично
\begin{align*}
xux^{-1}&=(a+bv)u(a-bv)= (a+bv)(au-buv])=\\
&=a^2u -ab[u,v]+ab[v,u]- b^2vuv=a^2u-2ab[u,v]-b^2\|v\|^2u=\\ &=(a^2-b^2)u-2ab[u,v]=\cos2\ffi u+ \sin 2\ffi [u,v]
\\
\\
x[u,v]x^{-1}&=(a+bv)[u,v](a-bv)= (a[u,v]+bu)(a-bv)=\\
&=a^2[u,v]+abu-ab[u,v]v-b^2uv=\\
&=(a^2-b^2)[u,v]+2abu=\cos 2\ffi[u,v]+\sin 2\ffi u
\end{align*}

Осталось показать, что только $x=\pm 1$ лежит в ядре этого отображения. Это возможно только тогда, когда $2\ffi \equiv 0 \mod 2\pi$. Значит $\ffi=2 \pi$ или $\ffi= \pi$. Первое соответствует $x=1$. Втораое --  $a=\cos \ffi = -1$, а $b=\sin \ffi = 0$, то есть $x=-1$. 
\endproof


\zd
Покажите, что отображение $(x,y) \to (z \to xzy^{-1})$ задаёт сюръективный гомоморфизм из декартового квадрата группы единичных кватернионов в группу $\SO_4(\mb R)$ с ядром $\{(1,1),(-1,-1)\}$.
\ezd

Обсудим теперь для чего могут понадобится кватернионы. Каждый кватернион, как мы установили кодирует вращение трёхмерного пространства или, что тоже самое -- новую декартову систему координат в $\mb R^3$ с центром в нуле. Такой тип данных встречается в компьютерной графике, если вы хотите зафиксировать ракурс, в котором вы смотрите на 3d-сцену или положение какого-то конкретного объекта в такой сцене.

В такой задаче кватернионы сложно превзойти. Действительно, задание сцены при помощи кватернионов очень экономно -- 4 коэффициента с одним соотношением на них (3 коэффициента, если очень нужно) против 9 коэффициентов у ортогональной матрицы.

А что с эффективностью операций? Тут вопрос состоит в том, про какие операции идёт речь. Если вы хотите сказать, что тот или иной вектор, который был повёрнут на кватернион $q=a+v$ надо повернуть ещё на кватернион $p=b+u$, то вам надо всего лишь вычислить $pq=ab+au+bv+uv$. Такое произведение считается за 16 умножений и 12 сложений. Если брать произведение матриц $3\times 3$, то там получается 27 умножений и 18 сложений (можно обойтись и 23 умножениями, но сильно увеличив число сложений).

Можно конечно использовать углы Эйлера, но тогда придётся использовать для вычислений косинусы и синусы этих углов, которые сами даются не бесплатно.

Если вам даны два ракурса в виде кватернионов, то легко понять, на какой кватернион надо домножить, чтобы из первого получить второй.

Но что если вы хотите просто повернуть при помощи кватерниона какой-то вектор из $\mb R^3$. Тут ситуация несколько хуже. Для этого вам необходимо посчитать $qxq^{-1}$. Предположив, что $q$ нормирован можно заменить обращение на сопряжение. Тем не менее такой подход довольно дорог -- 32 умножения и 24 сложения. Конечно, такой способ не оптимален. Например, не обязательно считать скалярную часть -- она должна стать нулевой. На самом деле, если $q=a+v$, то 
$$qxq^{-1}=x+ 2[v, [v,x]+ ax]$$
что даёт 15 умножений и 15 сложений (если считать умножение на 2 сложением). Это конечно отличается от матриц, где необходимо 9 умножений и 6 сложений.

Впрочем, отличается не сильно. Кроме того у кватернионного представления есть плюс произведение нескольких кватернионов -- это кватернион несмотря на ошибки округления. Что не так для ортогональных матриц.

Наконец, представим себе задачу, что нам нужно плавно перейти от ракурса $q$ к ракурсу $p$. Желательно с "равномерной" скоростью. На языке кватернионов это становится понятно. Для этого заметим, что единичные кватернионы -- это всего лишь точки на трёхмерной сфере. Несложно понять, что их соединяет часть дуги сферы, точки на которой заданы как 
$$ \frac{\sin(t\theta)p + \sin ((1-t)\theta) q}{\sin \theta},$$
где $\theta$ -- угол между $p$ и $q$ (острый, не забываем, что представление кватернионами немного не однозначно).

\end{document}
